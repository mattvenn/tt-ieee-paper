\section{Circuit Boards}
\label{sec:circuit_board}
After manufacture, the chips are mounted onto small carrier boards with \(0.1\) inch headers. This allows people with limited equipment or surface mount technology (SMT) assembly experience to build their own demonstration boards.

%\begin{figure}[htp]
%\centering
%\includegraphics[width=0.5\textwidth]{carrier_board_with_TT02}
%\caption{Carrier board with TT02 chip mounted.}
%\label{fig:carrier_board_with_TT02}
%\end{figure}

The carrier fits onto the demonstration board which provides:
\begin{itemize}
\item USB-C for power connection,
\item \(1.8v\) and \(3.3v\) power supplies for core and IO,
\item \(20MHz\) oscillator,
\item buttons for reset and single-step clock,
\item an 8-way DIP switch for inputs,
\item a 9-way DIP switch for design selection,
\item a 7-segment LED display for the outputs,
\item headers for all IO, including 2 standard Digilent ports (PMOD),
\item a header to select internal or external clock,
\item a header to select internal or external scan chain driver,
\item a header to engage an automatic clock divider in input pin 0.
\end{itemize}

\begin{figure}[htp]
\centering
\includegraphics[width=0.5\textwidth]{./Figs/tt02 board running design.JPG}
\caption{The demonstration board. Certified Open Source Hardware ES000040\cite{oshwacertification}.}
\label{fig:demonstration_board}
\end{figure}

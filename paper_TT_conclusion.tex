\section{Conclusion}
\label{sec:conclusion}

Tiny Tapeout runs 1 through 6 inclusive have demonstrated demand for more accessible entry points into chip design and fabrication. By lowering both the technical and financial barriers to entry, Tiny Tapeout is enabling hobbyists, educators, and others traditionally excluded to design and produce their own ASICs and to receive physical hardware in return.

Key to the success of the project is the inclusion of all projects from a given run on a single chip, going a step beyond the multi project wafer approach common in the industry today. By fabricating every design from a given run on every chip produced the cost can be shared between all participants, while the educational benefits are increased by allowing participants to experiment with others' designs in addition to their own.

Results from Tiny Tapeout runs for which hardware is available have proven the concept, and issues highlighted in earlier runs have been addressed for subsequent runs. The addition of analog and mixed circuit capabilities in Tiny Tapeout 6, the latest run at the time of writing, opens a new front for experimentation and education.

Community engagement in Tiny Tapeout has been strong with \qty{756} designs submitted over the first five shuttles. An online chat server for participants has \qty{1200} members and there are \qty{2600} subscribers to the project's mailing list.

The number of participants identifying as hobbyists, students, and teachers shows a demand for accessible chip design and manufacturing capabilities from groups largely overlooked by traditional approaches. More recent Tiny Tapeout runs have seen an increase in academic participants, from community colleges to universities, and we expect to see further growth from this sector as educators seek to deliver on industry demand for graduates with semiconductor design experience.

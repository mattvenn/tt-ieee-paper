\section{Introduction}
\label{sec:introduction}
\IEEEPARstart{T}{iniTapeout}
TinyTapeout~\cite{tinytapeout} is an educational project that makes it easier and cheaper than ever to get ASIC designs manufactured.
The digital design flow consists of templating a GitHub~\cite{github} repository, adding a design, waiting for the tests and binary layout files (GDS~\cite{gds}) generation to complete, then submitting to a quarterly shuttle.

Up to 500 designs are multiplexed to 24 general purpose input/output (GPIO) pins, and after manufacture the chips are mounted to a demonstration board for easy testing.
Each design can be activated and tested in turn.
Documentation submitted with each project forms a printable datasheet~\cite{datasheet} as well as an online index at TinyTapeout.com/runs/~\cite{tinytapeoutruns}

Design entry is done mostly with Verilog or Wokwi~\cite{wokwi}.
Wokwi is a schematic based editor that is an easy way to get started for people with no prior hardware description language (HDL) experience.
The TinyTapeout website includes a basic getting started guide for drawing circuits with Wokwi available in English and Spanish.

The first~\cite{firstshuttle} free and experimental shuttle with 152 designs was submitted to the seventh Google sponsored~\cite{googlesponsored} lottery multi project wafer (MPW) shuttle in September 2022.
The next 4 shuttles combined 582 designs and were sponsored by and manufactured with the Efabless~\cite{efabless} chipIgnite MPW service.

\begin{table*}[htbp]
\centering
\caption{Table shows the stats for each of the TinyTapeout shuttle runs.}
\label{tab:tinytapeout}
\begin{tabularx}{\textwidth}{@{}l *{6}{X}@{}}
\toprule
\textbf{Run} & \textbf{Launched} & \textbf{Closed} & \textbf{Shuttle} & \textbf{Designs} & \textbf{Chips Expected} & \textbf{Estimated delivery date} \\
\midrule
TT01 & 2022-08-17 & 2022-09-01 & MPW7  & 152 & 2024-01-30 & Not expecting to ship this test \\
TT02 & 2022-11-09 & 2022-12-02 & 2211Q & 165 & 2023-10-17 & 2024-01-30 \\
TT03 & 2023-03-01 & 2023-04-23 & 2304C & 249* & 2024-01-15 & 2024-02-28 \\
TT04 & 2023-07-01 & 2023-09-08 & 2309  & 143 & 2024-02-28 & 2024-04-15 \\
TT05 & 2023-09-11 & 2023-11-04 & 2311  & 174 & 2024-04-12 & 2024-05-12 \\
TT06 & 2024-02-01 & 2024-04-19 & 2404  & TBD & TBD        & TBD \\
\bottomrule
\end{tabularx}
\end{table*}

Each tile is approximately $100\times 160 um$, enough for around 1000 logic gates and is priced at \$50.
The physical chip and demo board are optional and cost an additional \$250.
Individuals pay a reduced \$100 for their first chip and board thanks to sponsorship by Efabless\cite{efabless}.

By separating the cost of area and the cost of the chip, a group of 10 could submit 10 designs and share 1 board for \$600.

The GitHub templates\cite{verilogtemplate} make use of GitHub Actions\cite{githubactions} - an automatic continuous integration system that is triggered every time the repository is updated.
There are 4 main jobs:

\begin{enumerate}
	\item GDS - installs OpenLane and the Sky130 process design kit (PDK), then builds the GDS, and generates a summary of the design that includes utilization, standard cells used, and a 2D and 3D model of the GDS.
This job can optionally also run a gate-level verification.
	\item Verification - installs the YosysHQ open source CAD suite which includes many common electronic design automation (EDA) tools.
Then iVerilog and cocotb are used to run any testbenches included.
	\item Documentation - generates a preview of the documentation.
	\item Precheck - a number of tests are run to make sure that the design doesn’t cause design rule check (DRC) errors after integration into the chip.
\end{enumerate}

Successful GDS, Documentation, and Precheck jobs are required to submit to a shuttle.
Verification is optional but highly encouraged. Wokwi designs can make use of an integrated truth table testing system\cite{automatedtesting}.

\begin{figure}[H]
\centering
	%\vspace{-0.5cm}
	\includegraphics[width=\columnwidth]{./Figs/gh action gds layout.png}
	%\vspace{-1.0cm}
	\caption{Top View 2D from}
	\label{fig:top_view_gds}
	%\vspace{-0.2cm}
\end{figure}

\begin{table*}[htbp]
\centering
\caption{Summary of the GDS job.}
\label{tab:gds_summary}
\begin{tabularx}{\textwidth}{@{}lXr@{}}
\toprule
\textbf{Category} & \textbf{Cells} & \textbf{Count} \\
\midrule
Fill & decap fill & 1145 \\
Combo Logic & o21ai and3b a221o o31a or2b a21o nobr2 o31ai a41o a21o a211o o21ba o21a o21bai a2bb2o or3b o221a a31o a22o o32a o22a a32o a2111o and2b and3b or4b or4bb o211a a2111oi o2bb2a a211oi and4bb o211ai a22oi a31oi o311a & 249 \\
Tap & tappwvrgnd & 246 \\
Flip Flops & dftxp & 146 \\
Buffer & buf clkbui & 127 \\
AND & and2 a1boi and3 and4 & 97 \\
Misc & dlygate4sd3 dlymetal6s2s comb & 84 \\
OR & or3 xor2 or2 or4 & 81 \\
NOR & xnor2 nor2 nor3 nor4 & 64 \\
NAND & nand2 nand3 nand4 nand2b & 52 \\
Inverter & inv & 37 \\
Multiplexer & mux2 mux4 & 9 \\
Diode & diode & 1 \\
\bottomrule
\end{tabularx}
\end{table*}

While the process can be done entirely in the browser, it’s also possible to install a local copy of the tools, which can help to reduce iteration time, especially for tests and verification.

Community engagement has been strong with 756 designs submitted over the 5 shuttles.
The Discord community has 1000 members with 1600 subscribed to the mailing list.

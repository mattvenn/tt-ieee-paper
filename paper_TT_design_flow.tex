\section{Design Flow}
\label{sec:design_flow}

Tiny Tapeout designs are primarily developed in the Verilog Hardware Description Language (HDL) or Wokwi~\cite{wokwi}.
Wokwi is a web based visual schematic editor for hardware description, designed as an easier way for individuals with no prior HDL experience to get started.
The TinyTapeout website~\cite{tinytapeout} includes a basic Wokwi getting started guide, demonstrating how to use the tool to draw circuits, which is made available in English and Spanish.

The design flow has the participant create a GitHub~\cite{github} source code repository based on provided templates then add their ASIC design. This triggers automated tests and the generation of binary layout files in GDSII~\cite{gds}. If all tests pass and the binary layout files are correctly generated, the design is then submitted to a quarterly shuttle for production in silicon.

The TinyTapeout GitHub templates\cite{verilogtemplate} make use of GitHub Actions\cite{githubactions}---an automatic continuous integration system triggered every time the repository is updated. This reduces duplicated effort and makes it possible for TinyTapeout to support large numbers of participants without excessive technical overhead.

There are four main jobs in the continuous integration system:

\begin{enumerate}
	\item GDS: installs OpenLane\cite{openlane} and the SkyWater Sky130\cite{skywaterpdk} PDK, builds the binary layout files, and generates a summary of the design (Fig.~\ref{fig:summary_table_GDS_job}). The summary includes utilization, standard cells used, a 2-D render (Fig.~\ref{fig:render_cells_in_use}) and an interactive 3-D viewer (Fig.~\ref{fig:interactive_3D_viewer}).
This job can also optionally run a gate-level verification of the design.
	\item Verification: installs the YosysHQ open source Computer-Aided Design (CAD) suite, which includes many common electronic design automation (EDA) tools; uses iVerilog\cite{iverilog} and cocotb\cite{cocotb} to run included testbenches.
	\item Documentation: generates a preview of the documentation.
	\item Precheck: runs Design Rule Check (DRC) tests to ensure the design can be integrated into the multi project chip.
\end{enumerate}

Successful GDS, Documentation, and Precheck job completion are all required for a design to be submitted to a shuttle for production.
Verification is optional but highly encouraged. Submissions designed in Wokwi are able to make use of its integrated truth table testing system\cite{automatedtesting}.

While the TinyTapeout continuous integration system can be run entirely in the user's web browser, it is also possible to install a local copy of the tools\cite{localinstall} on a participant's computer. Locally installed tools can help to reduce the time between design iterations, especially for the test and verification jobs.

\begin{figure}[!t]
\centering
\includegraphics[width=\columnwidth]{./Figs/gh action cell stats.png}
\caption{A summary table from the GDS continuous integration job.}
\label{fig:summary_table_GDS_job}
\end{figure}

\begin{figure}[!t]
\centering
\includegraphics[width=\columnwidth]{./Figs/gh action gds 3d view.png}
\caption{The interactive 3-D viewer.}
\label{fig:interactive_3D_viewer}
\end{figure}
